\documentclass[brudnopis]{xmgr}
% Jeśli nowe rozdziały mają się zaczynać na stronach
% nieparzystych:
%\documentclass[openright]{xmgr}

%\defaultfontfeatures{Scale=MatchLowercase}
%\setmainfont[Numbers=OldStyle,Ligatures=TeX]{Minion Pro}
%\setsansfont[Numbers=OldStyle,Ligatures=TeX]{Myriad Pro}
% for fontspec version < 2.0
\setmainfont[Numbers=OldStyle,Mapping=tex-text]{Minion Pro}
\setsansfont[Numbers=OldStyle,Mapping=tex-text]{Myriad Pro}
%\setmonofont[Scale=0.75]{Monaco}

% Opcjonalnie identyfikator dokumentu
% drukowany tylko z włączoną opcją 'brudnopis':
\wersja   {wersja wstępna [\ymdtoday]}

\author   {Marcin Dawidowski}
\nralbumu {231010}
\email    {marcin.dawidowskipl@gmail.com}

\title    {Let’s Bid It – portal aukcyjny}
\date     {2017}
\miejsce  {Gdańsk}

\opiekun  {dr Włodzimierz Bzyl}

% dodatkowe polecenia
%\renewcommand{\filename}[1]{\texttt{#1}}
%\definecolor{stress}{cmyk}{0,1,0.13,0} % RubineRed
%\definecolor{topic}{cmyk}{0.98,0.13,0,0.43} % MidnightBlue

\begin{document}

% streszczenie
\begin{abstract}
  W pracy przedstawiono wersję beta aplikacji webowej „Let's Bid It” do tworzenia i publikowania aukcji.
  
  W aplikacji zaimplementowano kategoryzację aukcji i zaawansowane wyszukiwanie ich za pomocą formularza oraz tworzenie kategorii w hierarchii.

  Zaprojektwano widok strony głównej wyświetlający losowe aukcje znajdujące się w portalu oraz panele administracyjny i użytkownika, widoczne po zalogowaniu się na konkretny typ konta. Każda z aukcji wyświetlana jest w kolejności rosnącej według czasu zakończenia.

  Do tworzenia aplikacji wykorzystano Ruby on Rails, a także Twitter Bootstrap do implementacji widoków. Użyto również gemu CKEditor pozwalający umieścić na stronie zaawansowany edytor tekstowy, a także Elasticsearch do zaawansowanego wyszukiwania aukcji na stronie. Do stworzenia drzewa kategorii użyto gemu Acts as tree.

\end{abstract}

% słowa kluczowe
\keywords{Ruby on Rails, gem, heroku, auction, online shopping, bid}

% tytuł i spis treści
\maketitle

% wstęp
\introduction

Chęć stworzenia czegoś od podstaw oraz własne doświadczenie w korzystaniu z portali aukcyjnych sprawiło, że dla mnie jako programisty stworzenie takiej aplikacji, byłoby niezwykle satysfakcjonujące i pozwoliłoby rozwinąć umiejętności. Ponadto stworzenie takiego portalu może ułatwić mi w przyszłości implementacje różnego rodzaju sklepów internetowych i innych serwisów związanych z handlem przez Internet.

Co udało się zrealizować?


\chapter{Wprowadzenie}

W dzisiejszych czasach bardzo wiele osób korzysta z portali aukcyjnych. 
Dzieję się tak przede wszystkim dlatego, że pomagają one zdecydowanie
zaoszczędzić czas, a także w dużej mierze pozwalają oszczędzić 
pieniądze, dzięki konkurencyjnym cenom oraz szerokiemu wahlarzowi dostępnego
towaru. Powstało już wiele stron, które pozwalają nam dokonywać zakupów w 
domowym zaciszu, jednak każda z nich posiada pewne cechy, które odróżniają
je od konkurencji. Tworząc tą aplikację miałem na celu wyłapać jak najwięcej 
tych cech i umieścić je w moim projekcie, który będzie je łączył.


\section{Porównanie dostępnych rozwiązań}

\begin{itemize}
\item Allegro - najpopularniejszy w Polsce portal aukcyjny, oferujący setki produktów w niezwykle
konkurencyjnych cenach. Portal udostępnia możliwość zakupu natychmiastowego, jak i licytacji
towaru. To jaki jest typ danej aukcji, zależy tylko i wyłącznie od jej właściciela.

\item OLX - jest to nie tyle portal aukcyjny, co portal ogłoszeniowy. Użytkownicy tego serwisu
mogą udostępniać ogłoszenia, jednak sama sprzedaż towaru przebiega już poza nim.

\item eBay - serwis bardzo popularny przede wszystkim poza granicami Polski. Portal ten niejako łączy
funkcje dwóch prędzej wymienionych stron. Użytkownicy mogą udostępnić towar do licytacji, zakupu 
w konkretnej cenie, lub też na zasadzie zwykłego ogłoszenia.
\end{itemize}


\section{Możliwości zastosowania praktycznego}
Aplikacja przeze mnie stworzona, poza tym, że może zostać użyta jako portal aukcyjny, może 
również znaleźć inne zastosowania w praktyce. 

Z niewielką pomocą, stworzona przeze mnie aplikacja, przekształcona może zostać również
w sklep internetowy, przedstawiający konkretną ofertę. Poza tym serwis ten mógłby znaleźć
również zastosowanie jako portal ogłoszeniowy. 

\chapter{Projekt i analiza}

\section{Diagram ERD}

\section{Aktorzy i przypadki użycia}

\section{Diagram klas}

\chapter{Implementacja}
W celu stworzenia aplikacji, która ma w odpowiedni sposób realizować postawione przed nią
założenia, potrzebne są odpowiednio dobrane technologie oraz narzędzia, dzięki którym uda się 
spełnić postawione przed nią wymagania. Dobór odpowiedniej technologii, a także frameworków 
jest kluczowy do skutecznej implementacji aplikacji. Poniżej omówię wybrane przeze mnie narzędzia.

\section{Ruby on Rails}
Do stworzenia projektu wykorzystana została technologia Ruby on Rails bazująca na języku Ruby.
W aplikacji wykorzystano Ruby w wersji 2.3, a także framework Rails w wersji 5.0.1. Całość stworzona
została przy pomocy architektury MVC (ang. Model-View-Controller).

\section{Twitter Bootstrap}
Widoki utworzone zostały przy pomocy frameworka Twitter Bootstrap, który pozwala na proste tworzenie
graficznego interfejsu stron internetowych z wykorzystaniem gotowych rozwiązań bazujących na językach
HTML i CSS. Główną zaletą tego frameworka jest responsywność, czyli zapewnienie dobrego wyświetlania
stron WWW na różnego typu urządzeniach. 

\section{Elasticsearch}
W celu skutecznego wyszukiwania danych na stronie wykorzystałem silnik Elasticsearch, który w dużym stopniu
potrafi ułatwić wyszukiwanie danych w bazie, zwłaszcza, gdy jest ich bardzo dużo. Sam silnik działa jako osobna 
aplikacja działająca w tle i pozwala wykorzystywać dużo bardziej zaawansowane metody wyszkiwania niż standardowe
metody języka Ruby.

\section{CKEditor}
Jest to wizualny edytor tekstowy języka HTML, który umożliwia użytkownikowi na wybranie między innymi konkretnej 
czcionki, jej rozmiaru, koloru liter, czy też ich stylu. Poza tym użytkownik może również ustawić wyrównanie tekstu,
a także możliwe jest wstawienie listy, tabeli, odnośników, a także obrazów.

\section{Act as tree}
Gem ten wykorzystany został w celu stworzenia drzewa kategorii, które ułatwia w sposób zdecydowany nawigację
po stronie, a także sprawia, że dane na stronie są uporządkowane. Sama implementacja przebiega w środowisku
technologii Ruby on Rails, więc nie wymaga bezpośredniej instalacji.

% zakończenie
\summary

% załączniki (opcjonalnie):
\appendix
\chapter{Tytuł załącznika jeden}

Treść załącznika jeden.

\chapter{Tytuł załącznika dwa}

Treść załącznika dwa.

% literatura (obowiązkowo):
\bibliographystyle{unsrt}
\bibliography{xml}

% spis tabel (jeżeli jest potrzebny):
\listoftables

% spis rysunków (jeżeli jest potrzebny):
\listoffigures

\oswiadczenie

\end{document}
